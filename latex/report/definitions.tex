\section{Definitions}
\label{sec:definitions}
In this section, we define the SQ learning model and the related terms. We begin by defining PAC learning since as mentioned previously SQ learning originates from it.

\begin{definition}[efficient PAC learning]
Let $C$ be a class of boolean functions $c: \cX \xrightarrow{} \{-1, 1\}$. Then $C$ is efficient PAC learnable if there exists an algorithm $\cA$ such that for every $c \in C$, any probability distribution $\cD$ over $X$, any $\epsilon > 0$, and any $\delta < 1$, $\cA$ takes a labeled sample $S$ of size $m = poly\left(\frac{1}{\epsilon}, \frac{1}{\delta}, |x|, |c|\right)$ from $\cD$ and in polynomial time $m$, outputs a hypothesis $h$ for which
    \begin{equation*}
        \Prob{err_\cD(h) \leq \epsilon} \geq 1 - \delta
    \end{equation*}
\end{definition}

\cite{angluin_learning_1988} extend the PAC learning model to handle random classification noise.

\begin{definition}[PAC learning with noise]
A PAC learning algorithm with random classification noise must meet the PAC requirements as before, but the labels of the sample $S$ of size $m = poly\left(\frac{1}{\epsilon}, \frac{1}{\delta}, |x|, |c|, \frac{1}{1-2\eta}\right)$ are flipped independently with probability $0 \leq \eta < \frac{1}{2}$.
\end{definition}

Now, we will first define the statistical query and the statistical query oracle.

\begin{definition}[statistical query]
A statistical query is a pair $(\chi, \tau)$ where $\chi: \cX \times \{-1, 1\} \xrightarrow{} \{-1, 1\}$ and $\tau \geq 0$ is the tolerance parameter.
\end{definition}

\begin{definition}[statistical query oracle]
The statistical query oracle is a function $SQ(c, \cD)$ which when given a function as input, calculates a statistical query using the function and returns a value in the range
\begin{equation*}
    [\mathbb{E}_{\cD}[\chi(x, c(x))] - \tau, \mathbb{E}_{\cD}[\chi(x, c(x))] + \tau]
\end{equation*}
\end{definition}

Finally, we can define efficient statistical query learning.

\begin{definition}[efficient SQ learning]
Let $C$ be a class of boolean functions $c: \cX \xrightarrow{} \{-1, 1\}$. Then $C$ is efficient SQ learnable if there exists an algorithm $\cA$ such that for every $c \in C$, any probability distribution $\cD$ over $\cX$, and any $\epsilon > 0$, there is a polynomial $p(\cdot, \cdot, \cdot)$ such that
\begin{enumerate}
    \item $\cA$ makes at most $p(\frac{1}{\epsilon}, |x|, |c|)$ calls to the SQ oracle
    \item the smallest $\tau$ that $\cA$ uses satisfies $\frac{1}{\tau} \leq p(\frac{1}{\epsilon}, |x|, |c|)$
    \item the statistical queries $\chi$ are evaluable in time $p(\frac{1}{\epsilon}, |x|, |c|)$
    \item $\cA$ outputs a hypothesis $h$ satisfying $err_\cD(h) \leq \epsilon$
\end{enumerate}
% $\cA$ takes a labeled sample $S$ of size $m = poly\left(\frac{1}{\epsilon}, \frac{1}{\delta}, |x|, |c|\right)$ from $D$ and in polynomial time $m$, outputs a hypothesis $h$ for which
    
\end{definition}

\subsection{Variants of SQ models}
\label{variants}
We can define the statistical query oracle differently to get extensions of the statistical query model. \citet{bshouty_using_2001} modified the oracle in order to take the pair of hypothesis function $h$ and tolerance parameter $\tau$ as input and output the estimate of correlation between the hypothesis and the target function. This correlational statistical query (CSQ) oracle is defined as

\begin{definition}[correlational statistical query oracle]
The correlational statistical query oracle is a function $CSQ(c, \cD)$ which when given a hypothesis function $h: X \xrightarrow{} \{-1, 1\}$ and a tolerance parameter $\tau$ as inputs returns any value in the range
\begin{equation*}
    [\mathbb{E}_{\cD}[h(x)c(x)] - \tau, \mathbb{E}_{\cD}[h(x)c(x)] + \tau]
\end{equation*}
\end{definition}

\cite{yang_learning_2001} present an ``honest" SQ (HSQ) oracle which takes two inputs a statistical query $\chi$ and the sample count $n$ and outputs the empirical average of the query by sampling the distribution $D$. The HSQ oracle is defined as

\begin{definition}[honest statistical query oracle]
The honest statistical query oracle is a function $HSQ(\chi, n)$ which when given a statistical query and the sample size $n$ as inputs returns the empiricial average
\begin{equation*}
    % [\EE{\chi(x, c(x))} - \tau, \EE{\chi(x, c(x))} + \tau]
    \frac{1}{n} \sum_{i=1}^n \chi(x_i, c(x_i))
\end{equation*}
\end{definition}
